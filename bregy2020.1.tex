\documentclass[conference]{IEEEtran}
\IEEEoverridecommandlockouts
% The preceding line is only needed to identify funding in the first footnote. If that is unneeded, please comment it out.
\usepackage{cite}
\usepackage{amsmath,amssymb,amsfonts}
\usepackage{algorithmic}
\usepackage{graphicx}
\usepackage{textcomp}
\usepackage{xcolor}
\def\BibTeX{{\rm B\kern-.05em{\sc i\kern-.025em b}\kern-.08em
    T\kern-.1667em\lower.7ex\hbox{E}\kern-.125emX}}
\begin{document}

\title{Modelos no tradicionales de computación y algunas de sus implementaciones hardware}

\author{\IEEEauthorblockN{Bregy Malpartida}
\IEEEauthorblockA{
UTEC - \textit{Universidad de ingeniería y tecnología}\\
Lima, Perú\\
bregy.malpartida@utec.edu.pe}
}

\maketitle

\begin{abstract}
Este artículo revisa tres modelos o paradigmas de computación, además de algunas implementaciones físicas de dichos modelos, específicamente: computación con autómatas celulares (Cellular Automata Computation), computación neuromórfica (Neuromorphic Computation) y computación basada en vida (Living Computation). Encontraremos relación entre estos modelos y veremos lo importante que es buscar el entendimiento de la computación y su naturaleza fundamental.  La finalidad del artículo es conseguir que el lector adquiera un somoero panorama de algunas de las diferentes vías que tenemos para seguir el modelamiento y el desarrollo de nuestros dispositivos de cómputo y cálculo, que cumplen un rol fundamental en el desarrollo tecnológico de nuestra especie.
\end{abstract}

\section{Introducción}
Los dispositivos de computación pueden ser vistos como una consecuencia de la necesidad de nuestra especie por interactuar con su medio, una forma de entender la tecnología humana es como la externalización que realiza el hombre de su memoria en su medio, es decir, una forma de extender sus capacidades perdurando información en medios externos (normalmente inorgánicos). El hardware computable se puede entender como una plataforma para el procesamiento de información. El procesamiento de dicha información puede llevarse a cabo por medios físicos (mecánicos, eléctricos, magnéticos, etc.) \cite{duchesneau2014computing}\cite{piccinini2010computation}. La naturaleza física de varios procesos computables nos permiten aprovechar dicha computación para poder robustecer el desarrollo y la usabilidad de nuestra tecnología. En este artículo se hará una revisión del desarrollo de hardware computable que se ha llevado a cabo en los últimos años, se partirá de los tiempos modernos y no se hará énfasis en la historia analógica del hardware para computación, teniendo como punto de referencia el año 1936, en el que A. Turing publicó su paper con un formalismo para computadora universal \cite{turing1936a}\cite{dodig2011significance}\cite{eberbach2004turing} que dio paso a la teoría de la computación moderna.

\subsection{¿Qué es computación?}
La definición de "computación" es bastante general, cuenta con muchos enfoques y tiene múltiples interpretaciones \cite{turing2009computing}. Sin embargo, para tener un punto de partida razonable podemos aceptar la veracidad de la tesis Church-Turing, donde todo modelo de computación posible y físicamente realizable no es más poderoso que una Máquina de Turing \cite{Ben-amram06thechurch-turing}\cite{eberbach2004turing}. Si esto es así, entonces, aunque tengamos muchas formas de expresar las cosas, en última instancia, podemos reducir cada "computación" a una Máquina de Turing, obteniendo la definición de computación cómo "todo aquello que podamos reducir a una Máquina de Turing". Si bien esta definición nos deja con más preguntas por responder, nos sirve para tener un marco formal desde donde partir, ya que el modelo de Máquina de Turing es un modelo lógico que ha sido consolidado con el paso del tiempo \cite{arora2009computational}.

\subsection{Sobre hardware para computación}
El hardware para computación se puede entender como un medio para el procesamiento/transformación de información. En un amplio sentido, la mayoría de herramientas que el humano construye sirven en la realización de la transformación de dicha información. A lo largo de la historia del hombre, dichas herramientas se han ido sofisticando hasta llegar a donde nos encontramos actualmente. Hemos sabido aprovechar la naturaleza de nuestro medio y así hemos conseguido sintetizar "la computación" en diferentes herramientas (piezas de tecnología) que nos sirven para seguir destilando nuestra tecnología. El alcance de este artículo se limita a describir los últimos modelos de hardware para computación digital.

\subsection{Computación digital}
La computación digital fue definida G. Stibitz como aquella computación, en donde la señal, como un voltaje en el tiempo, no es usada directamente para representar un valor, si no, para codificarlo \cite{dennhardt2014}. Las bases matemáticas de la computación digital se apoyan en el álgebra booleana \cite{shannon1938}. Los modelos de computación y hardware para computación que revisaremos en este artículo hacen uso de este tipo de computación (digital).

\section{Modelos estándar}
Es importante hablar de un modelo que ha sido el que ha dirigido el desarrollo de hardware para computación en los últimos años: La arquitectura de Von Neumann \cite{poznanovic2006}.

\subsection{La arquitectura Von Neumann}
La arquitectura de computación digital de Von Neumann es una arquitectura de máquina de computación que se caracteriza por tener dos etapas bien definidas y delimitadas: la unidad de proceso y control, y la unidad de memoria, ambas partes conectadas por un bus de datos, que además, tiene la capacidad de almacenar las instrucciones en su unidad de memoria \cite{poznanovic2006} \cite{neumann1945}. Este concepto ya había sido descrito por A. Turing en su modelo de Máquina Universal de Turing que plantea una máquina capaz de simular cualquier otra Máquina de Turing arbitraria con una entrada arbitraria \cite{turing2009computing}\cite{eberbach2004turing} \cite{aspray1990john}. Para conseguir esto, la máquina universal almacena en su memoria la descripción de la máquina a simular.
Este modelo ha sido una gran influencia en la línea que ha seguido el desarrollo de hardware para computación, pero también existen otros modelos que no siguen esta línea de desarrollo, y que son de los que se habla en este artículo.

\section{Cellular Automata Computation}
Un automata celular es un modelo computacional que puede ser descrito como un sistema digital con una red de autómatas finitos. El concepto de autómata celular ofrece un elegante modelo computacional concurrente de interacción local, en el que la tarea de computación es realizada a través de la evolución de su estado global \cite{dascalu2016cellular}. Esta característica concurrente y distribuida es totalmente diferente en estructura a los modelos clásicos, secuenciales y centralizados que han tenido mayor popularidad en los anteriores años \cite{poznanovic2006}. Von Neumann fue, de manera sorpresiva, uno de los pioneros de la computación y realizó un trabajo sobre máquinas auto-replicantes que inspiró muchos trabajos alrededor de estos mecanismos (auto-replicantes) \cite{poznanovic2006} \cite{chou1998self} \cite{aspray1990john}, estos trabajos promovieron el desarrollo de diversos modelos de autómatas celulares \cite{poznanovic2006}. Varios estudios teóricos  fueron realizados por S. Wolfram, que propuso, entre otras cosas, una clasificación fenomenológica a los distintos modelos de autómatas celulares \cite{wolfram2018}.

\subsection{Hardware para un autómata celular general}
La versión de hardware de un autómata celular debería de consistir en una red de máquinas de estado finito con un número máximo predefinido de estados, y un circuito lógico programable con un número de entradas que depende de la topología de interconexión \cite{poznanovic2006}. La construcción de un hardware que permita la operación de un autómata celular general debería tener las siguientes características \cite{poznanovic2006}: 

\begin{itemize}
\item Ser capaz de implementar cualquier regla local.
\item  Poder implementar topologías no-homogéneas teniendo vecindades no-homogéneas y reglas locales no-homogéneas.
\item Poder definir diferentes condiciones para el perímetro (en el modelo teórico es infinito).
\end{itemize}

La utilidad de este hardware yace en la capacidad de poner a prueba distintos modelos de autómata celular y así evaluar distintas aplicaciones en este modelo de computación. Un ejemplo de este trabajo es The BioWall, un hardware para autómata celular general que fue implementado teniendo en cuenta principios de generación embrionaria \cite{tempesti2002biowall}. Este proyecto dejó un framework sobre el que se pueden implementar distintos modelos de autómata celular cómo self-replicating loops \cite{tempesti2012}, constructores de Von Neumann \cite{chou1998self}, Turing Neural Networks \cite{poznanovic2006}, entre otros.

\subsection{Algunas aplicaciones específicas del hardware para autómata celular}
Las diferentes implementaciones reportadas en la literatura científica se encuentran como soluciones de hardware dedicadas a aplicaciones específicas y no como máquinas de cómputo general. Muchas de estas implementaciones son también realizadas en hardware reconfigurable como FPGAs \cite{poznanovic2006}.
Entre las aplicaciones más resaltantes se encuentran los modelos físicos como autómatas celulares de Greenberg-Hastings \cite{tamayohartman1995} o modelos de reacción-difusión \cite{adamatzky2005reaction}. Uno de los trabajos más recientes en las aplicaciones de autómatas celulares es un modelo que hibrida redes neuronales y autómatas celulares en un modelo diferenciable de morfogénesis \cite{mordvintsev2020growing}. Este trabajo es prometedor, ya que abre las puertas a una generalización en las aplicaciones autómatas celulares, y que además impactaría directamente en las aplicaciones de hardware para computar este tipo de modelos (autómatas celulares) dotándolos de aplicaciones para computación general.

\section{Neuromorphic Computation}
Computación Neuromórfica es un concepto creado por C. Med a finales de los años 80 \cite{chen2018neuromorphic} y que describe el uso de sistemas VLSI  que contienen componentes analógicos que emulan arquitecturas neurobiológicas presentes en nuestro sistema nervioso \cite{chen2018neuromorphic}. 
Este tipo de computación tiene como mayor cualidad el procesamiento paralelo y distribuido, esto se consigue al emular el comportamiento biológico de las neuronas. Específicamente, se busca distribuir el procesamiento y la memoria a través de diferentes unidades de cómputo usando un mecanismo llamado "neuronal signaling", este mecanismo requiere de que las unidades de procesamiento también tengan la capacidad de perdurar información, es decir que la unidad combinen procesamiento y memoria al mismo tiempo. 

\subsection{Los Memristores}
Los memristores son dispositivos electrónicos de memoria no volátil que fueron teorizados en 1971 por L. Chua como uno de los cuatro elementos electrónicos fundamentales de dos terminales después del resistor, el capacitor y el inductor \cite{chua1984nonlinear}. La mayor característica de estos dispositivos es su propiedad de alterar su resistencia modificando su voltaje y que este nuevo estado sea almacenado incluso perdiendo la alimentación eléctrica \cite{upadhyay2019emerging} \cite{jeong2016memristors}. Una de las limitantes actuales con estos dispositivos electrónicos es su implementabilidad física; aunque se han logrado avances \cite{jeong2016memristors}, aún tenemos límites para poder llevar esta tecnología a una escala de provecho.
Como ya se dijo anteriormente, una de las necesidades de la computación neuromórfica es que cada unidad de procesamiento (símil de neurona) tenga la capacidad, también, de perdurar un estado determinado (peso sináptico), es aquí en donde entran a tallar los memristores. Los memristores tienen características que lo convierten en un buen modelo de las sinapsis en el cerebro humano, ambos comparten la misma característica de conmutación: el ser capaz de modificar la eficiencia de la transferencia de las señales usando la influencia de la misma transferencia \cite{jeong2016memristors}\cite{sung2018perspective}. Recientemente, se ha desarrollado un nuevo dispositivo memristivo a base de capas de disulfuro de molibdeno y iones de litio entre ellos \cite{zhulilianglu2018ionic}. Dicho dispositivo tiene la capacidad de cambiar el ordenamiento de sus átomos de molibdeno en presencia o ausencia de átomos de litio \cite{zhulilianglu2018ionic}. Este dispositivo emula adecuadamente el comportamiento competitivo y cooperativo presente en las conexiones neuronales, esto lo hace un buen análogo de las sinapsis neuronales biológicas.


\section{Living Computation}
Living Computation es un concepto/paradigma de computación que busca entender a los organismos vivos como sistemas de computación de los cuales podemos extraer distintos mecanismos para conseguir una revolución en nuestros modelos computacionales \cite{aguilar2014thepast} En última instancia, se propone que la computación intrínseca en distintos organismos vivos es la que debería guiar nuestro desarrollo computacional y así conseguir reducir el gap entre el software y el hardware \cite{aguilar2014thepast}. Según D. Ackley, las formas en que la vida debería guiar nuestro diseño de computación son las siguientes:

\begin{itemize}
\item Entender a la computación como un proceso consecuencia de la interacción. Este es un enfoque totalmente distinto a la idea tradicional de computación algorítmica y procedural \cite{lynn1998what} \cite{wegner1997interaction}.
\item Entender a los programas (de computadora) como entidades físicas, estos ocupan actualmente un espacio físico, en la memoria RAM, el disco, o en cualquier otro dispositivo físico (media). Mientras un programa ocupa un determinado lugar físico, ningún otro puede existir en aquel medio físico, además cuando un programa está en ejecución, este consume a través de el CPU, es decir, como consecuencia de la instancia de programa.
\end{itemize}

En mi opinión, la vida opera bajo el mismo marco de computabilidad que opera cualquier otro sistema físico, en ese sentido, los organismos vivos son análogos directos a nuestras computadoras. La idea de asimilar a la computación como un fundamento de la naturaleza nos obliga a entender a todos los procesos computables como equivalentes. Esto último también lo sostiene S. Wolfram con su principio de equivalencia computacional \cite{wolfram2002new} \cite{flake1998computational}.

\section{Conclusiones}
Hemos revisado tres enfoques sobre computación no clásicos. Dichos modelos de computación, y sus respectivas implementaciones físicas, guardan una relación estrecha en que se promueve un cambio de paradigma en la idea de cómo concebimos la naturaleza de la computación. La finalidad de comprender a la computación y su naturaleza es conseguir interactuar de manera más profunda con nuestra realidad. El hombre está cada vez más cerca de crear su propia realidad y uno de los posibles requisitos para hacerlo es comprender el fundamento que subyace en nuestra actual, y ya existente, realidad, aquella realidad en la que estamos y con la que interactuamos.


% \section*{References}


\bibliographystyle{./bibliography/IEEEtran}
\bibliography{./bibliography/IEEEabrv,./bibliography/references}


% \vspace{12pt}
% \color{red}
% IEEE conference templates contain guidance text for composing and formatting conference papers. Please ensure that all template text is removed from your conference paper prior to submission to the conference. Failure to remove the template text from your paper may result in your paper not being published.

\end{document}
